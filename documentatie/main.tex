\documentclass[a4paper,12pt]{article}
\usepackage{graphicx} % Required for inserting images
\usepackage[T1]{fontenc}
\usepackage[utf8]{inputenc}
\usepackage[romanian]{babel}
\AddToHook{cmd/section/before}{\clearpage}
\usepackage{sectsty}
\usepackage{extsizes}
\usepackage{multicol,lipsum}
\usepackage
[
        a4paper,% other options: a3paper, a5paper, etc
        left=2cm,
        right=2cm,
        top=2.5cm,
        bottom=2.5cm
]
{geometry}
\usepackage{indentfirst}
\usepackage[indent]{parskip}

\begin{document}
    \begin{titlepage}
    \vspace*{\fill}
    \begin{center}
        \huge{\textbf{\uppercase{Managementul activitatilor sportive din cadrul UPT}}}
    \end{center}
    \vfill
    \begin{flushleft}
        \Large{
            \textbf{Candidat: Antonyo Topolniceanu}
            \newline
            \newline
            \textbf{Coordonator științific: Conf.dr.ing. Ciprian-Bogdan Chirilă}
        }
        \newline
        \newline
    \end{flushleft}
    \begin{center}
        \Large{
            Septembrie 2023
        }
    \end{center}
    \end{titlepage}

    \tableofcontents

    \section*{\huge\centering{\uppercase{Abstract}}}

    \section{\Large\centering\uppercase{Introducere}}

    \section{\Large\centering\uppercase{Tehnologii web}}

    \subsection{Java}
    Java a fost creat de către o echipă de cercetători condusă de către James
    Gosling, la Sun Microsystems, pentru a facilita comunicarea dintre dispozitivele
    electronice de consum. Edificarea limbajului de programare Java a început în anul 1991 și,
    în scurt timp, concentrarea echipei s-a schimbat pe o nouă nișă, World Wide Web.
    Java a fost lansat pentru prima dată în 1995, iar capacitățile acestuia de a oferi
    interactivitate și control multimedia a arătat că este deosebit de potrivit pentru Web.
    
    Până la sfârșitul anilor 1990, Java a adus multimedia pe internet și a început să se extindă
    dincolo de Web, alimentând dispozitive de consum, computere, chiar și computerul de bord al
    roverelor de explorare pe Marte ale NASA. Datorită popularității, Sun Microsystems a creat diferite
    varietăți de Java pentru diferite scopuri, inclusiv Java SE pentru computerele de consum, Java ME pentru 
    dispozitivele incorporate și Java EE pentru servere de internet și supercomputere.

    Diferența dintre modul de funcționare Java și alte limbaje de programare a fost 
    revoluționară. Codul în alte limbi este mai întăi tradus de un compilator în instrucțiuni pentru un anumit tip de computer. În schimb,
    compilatorul Java transformă codul în Bytecode, care este apoi interpretat de softwareul numit Java Runtime Environment
    (JRE) sau mașină virtuală Java. JRE acționează ca un computer virtual care interpretează Bytecode și îl traduce pentru
    computerul gazdă. Din acest motiv, codul Java poate fi scris în același mod pentru mai multe platforme 
    ("Write Once, Run Everywhere"), ceea ce a contribuit la popularitatea să pentru utilizarea pe Internet,
    unde multe tipuri diferite de computere pot prelua aceeași pagină Web.
    
    În ciuda asemănării numelor, limbajul JavaScript care a fost conceput pentru a rula în browserele web nu face parte din Java.
    Inițial se numea Mocha și apoi LiveScript înainte ca Netscape să primească o licență de marketing de la Sun Microsystems.
    \subsection{Javascript}
    Dezvoltat de Netscape în colaborare cu Sun Microsystems, JavaScript este una dintre cele mai populare tehnologii de bază ale web. De la
    începuturile sale, a fost o parte integrantă a aplicațiilor web, realizând interactivitatea și dinamica paginilor web.

    Începând cu 2023, 98.7\% dintre site-uri folosesc Javascript în partea clientului pentru prelucrarea comportamentului
    al paginii web, deseori încorporând librării third-party. Toate browserele web majore au un interpretor JavaScript dedicat pentru a executa
    codul pe dispozitivele utilizatorilor.
    
    Este un limbaj de nivel înalt, adesea compilat just-in-time, care se conformează standardului ECMAScript. Este un limbaj de programare orientat de obiecte
    bazate pe prototip și funcții de primă clasă. Include tipuri dinamice. Este multi-paradigmă și acceptă stiluri de programare bazate pe
    evenimente funcționale și imperative. Are interfețe de programare a aplicațiilor (API-uri) pentru lucrul cu text, date, expresii regulate, 
    structuri de date și manipulare de DOM (Document Object Model).

    Interpretoarele JavaScript au fost utilizate inițial doar în browserele web, dar acum sunt componente de bază ale unor servere și ale unor
    varietăți de aplicații, cel mai popular sistem de rulare folosit în aceste scopuri fiind Node.js.
    
    Weak typing-ul limbajului de programare JavaScript poate creea foarte ușor erori. În alte limbaje, cum ar fi Java și C++, trebuie specificat
    în mod explicit tipul unui variabile. Cu toate acestea, în JavaScript, interpretorul deduce automat tipul de date
    al unei variabile pe baza valorii care îi este atribuită.

    Constrângerea agresivă de tip este o caracteristică a JavaScript care obligă diferitele tipuri de date să fie compatibile între ele. De exemplu
    dacă încercați să atribuiți un număr unei variabile de tip string, JavaScript va converti automatul numărul într-un string, indiferent de valoarea inițială.
    Prin urmare, acest lucru poate duce la rezultate neașteptate și poate fi dificil de depanat.

    Din cauza acestor deficiențe, pentru dezvoltarea de aplicații pe scară largă, Microsoft a creat TypeScript.
    \subsection{Typescript}
    Prin definiție, „TypeScript este JavaScript pentru dezvoltarea de aplicații scalabile”.
    
    TypeScript este un limbaj puternic tipizat, orientat pe obiecte, compilat. A fost proiectat de Anders Hejlsberg (designer de C\#) 
    la Microsoft. TypeScript este atât un limbaj, cât și un set de instrumente. TypeScript este un superset tip de JavaScript compilat 
    în JavaScript. Cu alte cuvinte, TypeScript este JavaScript plus câteva caracteristici suplimentare.
    
    A fost lansat publicului în octombrie 2012, cu versiunea 0.8, după doi ani de dezvoltare internă la Microsoft.
    La scurt timp după lansarea publică inițială, limbajul în sine a fost lăudat, dar s-a criticat lipsa suportului matur al multor IDE,
    în afară de Microsoft Visual Studio, care nu era disponibil pe Linux și OS X la acel moment. Din aprilie 2021, există suport
    în alte IDE-uri și editori de text, inclusiv Emacs, Vim, WebStorm, Atom și Visual Studio Code al Microsoft.
    
    TypeScript este portabil peste browsere, dispozitive și sisteme de operare. Poate rula în orice mediu în care rulează JavaScript. 
    Spre deosebire de omologii săi, TypeScript nu are nevoie de un VM dedicat sau de un mediu de rulare specific pentru a fi executat.

    Este superior celorlalți omologi ai săi, cum ar fi limbajele de programare CoffeeScript și Dart, într-un mod în care TypeScript
    este JavaScript extins. În schimb, limbaje precum Dart, CoffeeScript sunt limbi noi în sine și necesită un mediu de execuție specific limbii.

    JavaScript este un limbaj interpretat. Prin urmare, trebuie rulat pentru a testa dacă este valid. Înseamnă că scrieți toate codurile doar 
    pentru a nu găsi nicio ieșire, în cazul în care există o eroare. Prin urmare, trebuie să petreceți ore întregi încercând să găsiți erori în cod. 
    Transpilerul TypeScript oferă caracteristica de verificare a erorilor. TypeScript va compila codul și va genera erori de compilare, dacă găsește 
    erori de sintaxă. Acest lucru ajută la evidențierea erorilor înainte de rularea codului.

    JavaScript nu este puternic tipizat. TypeScript vine cu un sistem opțional de tipizare statică și inferență de tip prin TLS (Serviciul de limbaj TypeScript). 
    Tipul unei variabile, declarată fără tip, poate fi dedus de TLS pe baza valorii sale.
    \subsection{Angular}
    \subsection{Spring Boot}
    \subsection{Spring Data JPA}
    \subsection{PostgreSQL}
    

    \section{\Large\centering\uppercase{Proiectarea}}

    \section{\Large\centering\uppercase{Implementare}}

    \section{\Large\centering\uppercase{Testare si instalare}}

    \section{\Large\centering\uppercase{Concluzii}}

    \section*{\Large\centering{\uppercase{Bibliografie}}}

\end{document}
